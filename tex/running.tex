\section{Running the benchmark}

Running a benchmark workload involves a number of steps, including data import, driver configuration and workload execution.
The data import step involves loading the output of the data generator into the vendor database,
deploying the database and, optionally, performing a warm-up phase. 
The other steps are explained in the following sections.

\subsection{Driver Configuration} 
\label{sub:driver_configuration}

Before running a benchmark workload the workload driver~\cite{test_driver} must be configured, this involves the following steps:
\begin{enumerate}
	\item Programming: implement a vendor specific database connector (covered in section \ref{ssub:vendor_specific_database_connector})
	\item Configuration: configure the general driver properties (covered in section \ref{ssub:general_driver_properties})
	\item Configuration: configure the workload specific driver properties (covered in \ref{ssub:workload_specific_driver_properties})
\end{enumerate}

\subsubsection{Vendor specific database connector} 
\label{ssub:vendor_specific_database_connector}
Vendors must provide the workload driver with implementations for a number of Java classes: \texttt{Db} and \texttt{OperationHandler}.
More specifically, one implementation of \texttt{Db}, 
(optionally) one implementation of \texttt{DbConnectionState} (used to manage shared resources among queries, e.g. network connections), 
and one implementation of \texttt{OperationHandler} per query in the workload.

As a guide, to implement the necessary classes for a workload of two queries (\texttt{LdbcQuery1} and \texttt{LdbcQuery2}), 
see the code snippet below.
For a more detailed explanation of how this is done refer to the workload driver documentation \cite{test_driver}.

{\footnotesize
	\begin{verbatim}
    public class ExampleDbConnectionState extends DbConnectionState {
        private Client client;		
        public ExampleDbConnectionState(String url){
            client = new Client(url);
        }

        public getClient(){
            return client;
        }
    }

    public class ExampleDb extends Db {
        private ExampleDbConnectionState state;

        @Override
        protected void onInit(Map<String, String> properties) throws DbException {
            registerOperationHandler(LdbcQuery1.class, LdbcQuery1Handler.class);
            registerOperationHandler(LdbcQuery2.class, LdbcQuery2Handler.class);
            state = new ExampleDbConnectionState(properties.get(``url''));
        }

        @Override
        protected void onCleanup() throws DbException {}

        @Override
        protected DbConnectionState getConnectionState() throws DbException {
            return state;
        }

        public static class LdbcQuery1Handler extends OperationHandler<LdbcQuery1> {
            @Override
            protected OperationResult executeOperation(LdbcQuery1 operation) throws DbException {
                Client client = ((ExampleDbConnectionState)dbConnectionState()).getClient();
                LdbcQuery1Result result = client.execute(operation)
                int resultCode = // typically used for debugging, e.g. to return error codes
                return operation.buildResult(resultCode, result);
            }
        }

        public static class LdbcQuery2ToHandler extends OperationHandler<LdbcQuery2> {
            @Override
            protected OperationResult executeOperation(LdbcQuery2 operation) throws DbException {
                Client client = ((ExampleDbConnectionState)dbConnectionState()).getClient();
                LdbcQuery2Result result = client.execute(operation)
                int resultCode = // typically used for debugging, e.g. returning error codes
                return operation.buildResult(resultCode, result);
            }
        }
    }
	\end{verbatim}
}

\subsubsection{General driver properties}
\label{ssub:general_driver_properties}
Although a detailed explanation of the workload driver design is beyond the
scope of this document, it is worth noting that it has a number of
configuration parameters, these parameters are summarized in
\tref{tab:general_driver_parameters}.

\begin{table}[h!]
	\label{tab:general_driver_parameters}
	\begin{center}
		\begin{tabular}{|l|l|l|}
		\hline
		status 					& boolean & intermittently output workload status during execution \\
		\hline
		operationcount 			& integer & number of queries to generate \\
		\hline
		threadcount 			& integer & size of thread pool to use for query execution \\
		\hline
		resultfile 				& string & path to where workload results will be written \\
		\hline
		timeunit 				& enum & time unit to report result metrics in \\
		\hline
		timecompressionratio 	& double & adjust all query start times proportionally \\
		\hline
		gctdeltaduration 		& integer & min duration (ms) between dependent reads and writes \\
		\hline
		peeridentifiers 		& string[] & addresses of other driver processes (for distributed mode) \\
		\hline
		toleratedexecutiondelay & integer & max allowed time between scheduled and actual query start times \\
		\hline
		workload 				& string & class name of specific \texttt{Workload} implementation \\
		\hline
		database 				& string & class name of specific \texttt{Db} implementation \\  
		\hline
		\end{tabular}
	\end{center}
	\caption{General Driver Parameters}
\end{table}
Note that default configuration files, with relevant parameter values set and required parameters commented, 
are provided in the driver distribution.

\subsubsection{Workload specific driver properties}
\label{ssub:workload_specific_driver_properties}

In addition to the general driver properties, the driver supports the definition of arbitrary properties, 
which are passed along to the pluggable driver components: \texttt{Db} and \texttt{Workload}.
The first, \texttt{Db}, was described already in the previous section.
The second is \texttt{Workload}, it is the class responsible for defining which operations, in which order and at what throughput, the driver will generate.
From the point of view of the driver, the LDBC Social Network Benchmark is an implementation of the \texttt{Workload} class,
in this case named \texttt{LdbcInteractiveWorkload}. 
For more details regarding the workload itself see \ref{section:workload}.

The configuration parameters for the \texttt{LdbcInteractiveWorkload} workload are summarized in \tref{tab:snb_interactive_driver_parameters}.

\begin{table}[h!]
	\label{tab:snb_interactive_driver_parameters}
	\begin{center}
		\begin{tabular}{|l|l|l|}
		\hline
		parameter\_dir &
		string &
		data generator parameters directory (read parameters) \\
		\hline
		data\_dir &
		string &
		data generator data directory (dataset and write parameters) \\
		\hline
		LdbcQuery1\_interleave &
		boolean &
		interval between successive query 1 executions\\
		\hline
		LdbcQuery2\_interleave &
		boolean &
		interval between successive query 2 executions\\
		\hline
		LdbcQuery3\_interleave &
		boolean &
		interval between successive query 3 executions\\
		\hline
		LdbcQuery4\_interleave &
		boolean &
		interval between successive query 4 executions\\
		\hline
		LdbcQuery5\_interleave &
		boolean &
		interval between successive query 5 executions\\
		\hline
		LdbcQuery6\_interleave &
		boolean &
		interval between successive query 6 executions\\
		\hline
		LdbcQuery7\_interleave &
		boolean &
		interval between successive query 7 executions\\
		\hline
		LdbcQuery8\_interleave &
		boolean &
		interval between successive query 8 executions\\
		\hline
		LdbcQuery9\_interleave &
		boolean &
		interval between successive query 9 executions\\
		\hline
		LdbcQuery10\_interleave &
		boolean &
		interval between successive query 10 executions\\
		\hline
		LdbcQuery11\_interleave &
		boolean &
		interval between successive query 11 executions\\
		\hline
		LdbcQuery12\_interleave &
		boolean &
		interval between successive query 12 executions\\
		\hline
		LdbcQuery13\_interleave &
		boolean &
		interval between successive query 13 executions\\
		\hline
		LdbcQuery14\_interleave &
		integer &
		interval between successive query 14 executions\\
		\hline
		LdbcQuery1\_enable &
		boolean &
		enable/disable read query 1 \\
		\hline
		LdbcQuery2\_enable &
		boolean &
		enable/disable read query 2 \\
		\hline
		LdbcQuery3\_enable &
		boolean &
		enable/disable read query 3 \\
		\hline
		LdbcQuery4\_enable &
		boolean &
		enable/disable read query 4 \\
		\hline
		LdbcQuery5\_enable &
		boolean &
		enable/disable read query 5 \\
		\hline
		LdbcQuery6\_enable &
		boolean &
		enable/disable read query 6 \\
		\hline
		LdbcQuery7\_enable &
		boolean &
		enable/disable read query 7 \\
		\hline
		LdbcQuery8\_enable &
		boolean &
		enable/disable read query 8 \\
		\hline
		LdbcQuery9\_enable &
		boolean &
		enable/disable read query 9 \\
		\hline
		LdbcQuery10\_enable &
		boolean &
		enable/disable read query 10 \\
		\hline
		LdbcQuery11\_enable &
		boolean &
		enable/disable read query 11 \\
		\hline
		LdbcQuery12\_enable &
		boolean &
		enable/disable read query 12 \\
		\hline
		LdbcQuery13\_enable &
		boolean &
		enable/disable read query 13 \\
		\hline
		LdbcQuery14\_enable &
		boolean &
		enable/disable read query 14 \\
		\hline
		LdbcUpdate1AddPerson\_enable &
		boolean &
		enable/disable update query 1 \\
		\hline
		LdbcUpdate2AddPostLike\_enable &
		boolean &
		enable/disable update query 2 \\
		\hline
		LdbcUpdate3AddCommentLike\_enable &
		boolean &
		enable/disable update query 3 \\
		\hline
		LdbcUpdate4AddForum\_enable &
		boolean &
		enable/disable update query 4 \\
		\hline
		LdbcUpdate5AddForumMembership\_enable &
		boolean &
		enable/disable update query 5 \\
		\hline
		LdbcUpdate6AddPost\_enablet &
		boolean &
		enable/disable update query 6 \\
		\hline
		LdbcUpdate7AddComment\_enable &
		boolean &
		enable/disable update query 7 \\
		\hline
		LdbcUpdate8AddFriendship\_enable &
		boolean &
		enable/disable update query 8 \\
		\hline
		\end{tabular}
	\end{center}
	\caption{LDBC Social Network Benchmark Parameters}
\end{table}

\subsection{Running Workload} 
\label{sub:running_workload}

Finally, to run the workload execute the \texttt{main} method of the \texttt{Client} class in the driver.
The general pattern for doing so is as follows:

{\footnotesize
	\begin{verbatim}
	usage: java -cp core-VERSION.jar com.ldbc.driver.Client [-db <classname>] [-del <duration>] [-gctd <duration>]
	       [-oc <count>] [-P <file1:file2>] [-p <key=value>] [-pids <peerId1:peerId2>] [-rf <path>] [-s] [-tc
	       <count>] [-tcr <ratio>] [-tu <unit>] [-w <classname>]
	   -db,--database <classname>                    classname of the DB to use (e.g.
	                                                 com.ldbc.driver.workloads.simple.db.BasicDb)
	   -del,--toleratedexecutiondelay <duration>     duration (ms) an operation handler may miss its scheduled
	                                                 start time by
	   -gctd,--gctdeltaduration <duration>           safe duration (ms) between dependent operations
	   -oc,--operationcount <count>                  number of operations to execute (default: 0)
	   -P <file1:file2>                              load properties from file(s) - files will be loaded in the
	                                                 order provided
	                                                 first files are highest priority; later values will not
	                                                 override earlier values
	   -p <key=value>                                properties to be passed to DB and Workload - these will
	                                                 override properties loaded from files
	   -pids,--peeridentifiers <peerId1:peerId2>     identifiers/addresses of other driver workers (for
	                                                 distributed mode)
	   -rf,--resultfile <path>                       where benchmark results JSON file will be written (null =
	                                                 file will not be created)
	   -s,--status                                   show status during run
	   -tc,--threadcount <count>                     number of worker threads to execute with (default: 2)
	   -tcr,--timecompressionratio <ratio>           change duration between operations of workload
	   -tu,--timeunit <unit>                         time unit to use when gathering metrics.
	                                                 default:MILLISECONDS, valid:[NANOSECONDS, MICROSECONDS,
	                                                 MILLISECONDS, SECONDS, MINUTES]
	   -w,--workload <classname>                     classname of the Workload to use (e.g.
	                                                 com.ldbc.driver.workloads.simple.SimpleWorkload)
	\end{verbatim}
}

For the Interactive workload of LDBC-SNB it is necessary to provide the driver with four configuration files, containing:
general driver properties,
workload specific driver properties (e.g. database connection string),
a dataset specific value for one of the general driver properties (specifically, \texttt{gctdeltaduration}),
and vendor specific database properties.
The commandline to execute the workload would look something like:

{\footnotesize
	\begin{verbatim}
	java -cp ldbc_driver/target/core-0.2-SNAPSHOT.jar com.ldbc.driver.Client 
		-db com.vendor.VendorDb 
		-P vendor/vendor.properties,
		-P data_generator/outputDir/updateStream_0.properties,
		-P ldbc_driver/workloads/ldbc/socnet/interactive/ldbc_socnet_interactive.properties 
		-P ldbc_driver/src/main/resources/ldbc_driver_default.properties
	\end{verbatim}
}
