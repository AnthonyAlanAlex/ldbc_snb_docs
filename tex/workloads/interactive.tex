\subsection{Interactive Workload}
\subsubsection{Choke Points}

The design of the interactive workload queries has been conceived around two
main aspects: realism and technological relevance.  While realism has been
assessed by looking at existing social networks and thinking about what
interesting functionalities a user might desire from them, technological
relevance has been achieved by identifying a set of choke points queries should
stress.  These choke points capture those critical operations, techniques or
technologies that  could significatively affect the performance of the queries.
The choke points can be summarized in the following list:

\begin{itemize}
    \item \textbf{Aggregation Performance.}

The queries generally have a top k order by and often a group by in
addition to this.  These offer multiple optimization opportunities.
The queries also often have distinct operators, i.e. distinct friends
within two steps.  Collectively these are all set operations that may
be implemented with some combination of hash and sorting, possibly
exploiting ordering in the data itself.  The aggregates are not
limited to counts and sums.  For example string concatenation occurs
as an aggregate, testing possible user defined aggregate support.
There is a wide range of cardinalities in grouping, from low, e.g. country, to high, e.g. post.

    \item \textbf{Join Performance.}

Each graph traversal step is in principle a join.  The join patterns are
diverse, exercising both index and hash based operators.   Queries are designed
so as to reward judicious use of hash join by having patterns starting with one
entity, fetching many related entities and then testing how these are related
to a third entity, e.g. posts of a user with a tag of a given type. 

    \item \textbf{Data Access Locality.}

Graph problems are notoriously non-local.  However, when queries touch
any non-trivial fraction of a dataset, locality will emerge and can be
exploited, for example by vectored index access or by ordering data so
that that a merge join is possible.

    \item \textbf{Expression Calculation.}

Queries often have expressions, including conditional expressions.
This provides opportunities for vectoring and tests efficient
management of intermediate results.

    \item \textbf{Correlated Subqueries.}

The workload has many correlated subqueries, for example constructts
like x within two steps but not in one step, which would typically be
a correlated subquery with NOT EXISTS.  There are also scalar
subqueries with aggregation, for example returning the count of posts
satisfying a certain criteria.


    \item \textbf{Parallelism and Concurrency.}

All queries offer opportunities for parallelism.  This tests a wide
range of constructs, for example partitioned parallel variants of
group by and distinct.  An interactive workload will typically avoid
trivially parallelizable table scans.  Thus the opportunities that
exist must be captured by index based, navigational query plans.  The
choice of whether to parallelize or not is often left to run time and
will have to depend on the actual data seen in the execution, as
starting a parallel thread with too little to do is
counter-productive.


    \item \textbf{Graph Specifics.}

Graph problems are generally characterized by transitive properties
and the fact that neighboring vertices often have a large overlap in
their environments.  This makes cardinality estimation harder.  For
example, a query optimizer needs to recognize whether a relationship
has a tree or graph shape in order to make correct cardinality
estimations.  Further, there are problems aggregating properties over
a set of consecutive edges.  The workload contains business questions
dealing with paths and aggregates across paths, as well as the easier
case of determining a membership in a hierarchy with a transitive
part-of relation.
\end{itemize}


\subsubsection{Query Description Format}
\label{sub:queries_structure}
Queries are described in natural language using a well-defined structure that consists of three sections:
\textit{description}, a concise textual description of the query; 
\textit{parameters}, a list of input parameters and their types;
and \textit{results}, a list of expected results and their types.
The syntax used in \textit{parameters} and \textit{results} sections is as follows:

{\small
\begin{itemize}
	\item \textbf{Entity}: entity type in the dataset.\\
		One word, possibly constructed by appending multiple words together, starting with uppercase character and following the camel case notation,
		e.g. \texttt{TagClass} represents an entity of type ``TagClass''.
	\item \textbf{Relationship}: relationship type in the dataset.\\	
		One word, possibly constructed by appending multiple words together, starting with lowercase character and following the camel case notation,
		and surrounded by arrow to communicate direction, 
		e.g. \texttt{-worksAt->} represents a directed relationship of type ``worksAt''.
	\item \textbf{Attribute}: attribute of an entity or relationship in the dataset.\\
		One word, possibly constructed by appending multiple words together, starting with lowercase character and following the camel case notation,
		and prefixed by a ``.'' to dereference the entity/relationship, 
		e.g. \texttt{Person.firstName} refers to ``firstName'' attribute on the ``Person'' entity, 
		and \texttt{-studyAt->.classYear} refers to ``classYear'' attribute on the ``studyAt'' relationship.
	\item \textbf{Unordered Set}: an unordered collection of distinct elements.\\
		Surrounded by \{ and \} braces, with the element type between them,
		e.g. \texttt{\{String\}} refers to a set of strings.
	\item \textbf{Ordered List}: an ordered collection where duplicate elements are allowed.\\
		Surrounded by [ and ] braces, with the element type between them,
		e.g. \texttt{[String]} refers to a list of strings.
	\item \textbf{Ordered Tuple}: a fixed length, fixed order list of elements, where elements at each position of the tuple have predefined, possibly different, types. \\
		Surrounded by < and > braces, with the element types between them in a specific order
		e.g. \texttt{<String, Boolean>} refers to a 2-tuple containing a string value in the first element and a boolean value in the second,
		and \texttt{[<String, Boolean>]} is an ordered list of those 2-tuples.
\end{itemize}
}

\subsubsection{Lookup Query Descriptions}
\label{sub:queries}

Notes:
\begin{itemize}
\item Some queries require returning the content of a post. As stated in the schema, posts have content or imageFile, but not both. An empty string in content represents the post not having content, therefore, it must have a non empty string in imageFile and the other way around.
\end{itemize}

{\small
\begin{enumerate}
	\item Friends with certain name
	\begin{itemize}
		\item \textbf{Description:}
            Given a start Person, find up to 20 Persons with a given first name
            that the start Person is connected to (excluding start Person) by
            at most 3 steps via Knows relationships. Return Persons, including
            summaries of the Persons’ workplaces and places of study.  Sort
            results ascending by their distance from the start Person, for
            Persons within the same distance sort ascending by their last name,
            and for Persons with same last name ascending by their identifier
		\item \textbf{Parameters:} \\
			\begin{tabular}{ll}
				Person.id 										& ID \\
				Person.firstName								& String \\
			\end{tabular}
		\item \textbf{Results:} \\
			\begin{tabular}{ll}
				Person.id 										& ID \\
				Person.lastName									& String \\
				Person.birthday 								& Date \\
				Person.creationDate 							& DateTime  \\
				Person.gender 									& String \\
				Person.browserUsed 								& String \\
				Person.locationIP 								& String \\
				\{Person.emails\} 								& \{String\} \\
				\{Person.language\}  							& \{String\} \\
				Person-isLocatedIn->Place.name 				& String \\
				\{Person-studyAt->University.name, \\
				Person-studyAt->.classYear,  \\
				Person-studyAt->University-isLocatedIn->City.name\}	& \{<String, 32-bit Integer, String>\} \\
				\{Person-workAt->Company.name, \\
				Person-workAt->.workFrom, \\
				Person-workAt->Company-isLocatedIn->Country.name\} & \{<String, 32-bit Integer, String>\} \\
			\end{tabular}
	\end{itemize}

	\item Recent posts and comments by your friends
	\begin{itemize}
		\item \textbf{Description:}
			Given a start Person, find (most recent) Posts and Comments from all of that Person's friends, that were created before (and including) a given date. 
			Return the top 20 Posts/Comments, and the Person that created each of them.
			Sort results descending by creation date, and then ascending by Post identifier.			
		\item \textbf{Parameters:} \\
			\begin{tabular}{ll}
				Person.id 										& ID \\
				date 											& DateTime \\
			\end{tabular}
		\item \textbf{Results:} \\
			\begin{tabular}{ll}
				Person.id 										& ID \\
				Person.firstName								& String \\
				Person.lastName									& String \\
				Post.id/Comment.id 								& ID \\
				Post.content/Post.imageFile/Comment.content 	& String \\
				Post.creationDate/Comment.creationDate			& DateTime \\
			\end{tabular}		
	\end{itemize}

	\item Friends and friends of friends that have been to countries X and Y
	\begin{itemize}
		\item \textbf{Description:}
            Given a start Person, find Persons that are their friends and
            friends of friends (excluding start Person) that have made
            Posts/Comments in the given Countries X and Y within a given
            period.  Only Persons that are foreign to Countries X and Y are
            considered, that is Persons whose Location is not Country X or
            Country Y.  Return top 20 Persons, and their Post/Comment counts,
            in the given countries and period.  Sort results descending by
            total number of Posts/Comments, and then ascending by Person
            identifier.
		\item \textbf{Parameters:} \\
			\begin{tabular}{lll}
				Person.id 										& ID & \\
				CountryX.name									& String & \\
				CountryY.name									& String & \\
				startDate										& Date 	& // beginning of requested period \\
				duration										& 32-bit Integer 					& // duration of requested period, in days \\
                                                                &                                   & // the interval [startDate, startDate + Duration) is closed-open\\
			\end{tabular}		
		\item \textbf{Results:} \\
			\begin{tabular}{lll}
				Person.id 										& ID 	& \\
				Person.firstName 								& String 			& \\
				Person.lastName 								& String 			& \\
				countx 											& 32-bit Integer 	& \parbox[t]{20cm}{// number of Posts/Comments from Country X made by Person \par 
																					within the given time\strut} \\
				county 											& 32-bit Integer 	& \parbox[t]{20cm}{// number of Posts/Comments from Country Y made by Person \par 
																					within the given time\strut} \\
				count 											& 32-bit Integer 	& // countx + county \\
			\end{tabular}		
	\end{itemize}
		
	\item New topics
	\begin{itemize}
		\item \textbf{Description:}
			Given a start Person, find Tags that are attached to Posts that were created by that Person's friends.
			Only include Tags that were attached to Posts created within a given time interval, and that were never attached to Posts created before this interval.
			Return top 10 Tags, and the count of Posts, which were created within the given time interval, that this Tag was attached to.
			Sort results descending by Post count, and then ascending by Tag name.
		\item \textbf{Parameters:} \\
			\begin{tabular}{lll}
				Person.id 										& ID 	& \\
				startDate 										& Date & \\
				duration										& 32-bit Integer 					& // duration of requested period, in days \\
                                                                &                                   & // the interval [startDate, startDate + Duration) is closed-open\\
			\end{tabular} 
		\item \textbf{Results:} \\
			\begin{tabular}{lll}
				Tag.name 										& String 	& \\
				count 											& 32-bit Integer & // number of Posts made within the given time interval that have this Tag \\
			\end{tabular}		
	\end{itemize}

	\item New groups
	\begin{itemize}
		\item \textbf{Description:}
            Given a start Person, find the Forums which that Person's friends
            and friends of friends (excluding start Person) became Members of
            after a given date.  Return top 20 Forums, and the number of Posts
            in each Forum that was Created by any of these Persons - for each
            Forum consider only those Persons which joined that particular
            Forum after the given date.  Sort results descending by the count
            of Posts, and then ascending by Forum name
		\item \textbf{Parameters:} \\
			\begin{tabular}{lll}
				Person.id 										& ID & \\
				date 											& Date & \\
			\end{tabular}		
		\item \textbf{Results:} \\
			\begin{tabular}{lll}
				Forum.title 										& String & \\
				count 	 											& 32-bit Integer & \parbox[t]{20cm}{// number of Posts made in Forum that were created 																							by friends \par \strut} \\
			\end{tabular}		
	\end{itemize}

	\item Tag co-occurrence
	\begin{itemize}
		\item \textbf{Description:}
			Given a start Person and some Tag, find the other Tags that occur together with this Tag on Posts that were created by start Person's friends and friends of friends (excluding start Person).
			Return top 10 Tags, and the count of Posts that were created by these Persons, which contain both this Tag and the given Tag. Sort results descending by count, and then ascending by Tag name.
		\item \textbf{Parameters:} \\
			\begin{tabular}{lll}
				Person.id 										& ID & \\
				Tag.name 	 									& String & \parbox[t]{20cm}{\par \strut} \\
			\end{tabular}		
		\item \textbf{Results:} \\
			\begin{tabular}{lll}
				Tag.name 	 						& String & \parbox[t]{20cm}{\par \strut} \\
				count 								& 32-bit Integer & \parbox[t]{20cm}{// number of Posts that were created by friends and friends of friends, \par 
																							which contain this Tag\strut} \\
			\end{tabular}		
	\end{itemize}

	\item Recent likes
	\begin{itemize}
		\item \textbf{Description:}
             Given a start Person, find (most recent) Likes on any of start Person's Posts/Comments.
             Return top 20 Persons that Liked any of start Person's Posts/Comments, the Post/Comment they liked most recently, creation date of that Like, and the latency (in minutes) between creation of Post/Comment and Like.
             Additionally, return a flag indicating whether the liker is a friend of start Person.
             In the case that a Person Liked multiple Posts/Comments at the same time, return the Post/Comment with lowest identifier.
             Sort results descending by creation time of Like, then ascending by Person identifier of liker.
		\item \textbf{Parameters:} \\
			\begin{tabular}{lll}
				Person.id 	 						& 64-bit Integer & \parbox[t]{20cm}{\par \strut} \\
			\end{tabular}		
		\item \textbf{Results:} \\
			\begin{tabular}{lll}
				Person.id 	 								& ID & \parbox[t]{20cm}{\par \strut} \\
				Person.firstName 							& String & \parbox[t]{20cm}{\par \strut} \\
				Person.lastName 	 						& String & \parbox[t]{20cm}{\par \strut} \\
				Like.creationDate 	 						& DateTime & \parbox[t]{20cm}{\par \strut} \\
				Post.id/Comment.id 	 						& ID & \parbox[t]{20cm}{\par \strut} \\
				Post.content/Post.imageFile/Comment.content	& String & \parbox[t]{20cm}{\par \strut} \\
				latency 	 								& 32-bit Integer & \parbox[t]{20cm}{// duration between creation of\par Post/Comment and Like, in minutes\strut} \\
				isNew										& Boolean & \parbox[t]{20cm}{// false if liker Person is friend of\par start Person, true otherwise \strut} \\
			\end{tabular}		
	\end{itemize}

	\item Recent replies
	\begin{itemize}
		\item \textbf{Description:}
		\item \textbf{Parameters:} \\
			\begin{tabular}{lll}
				Person.id 	 						& ID & \parbox[t]{20cm}{\par \strut} \\
			\end{tabular}		
		\item \textbf{Results:} \\
			\begin{tabular}{lll}
				Person.id 	 				& ID & \parbox[t]{20cm}{\par \strut} \\
				Person.firstName 	 		& String & \parbox[t]{20cm}{\par \strut} \\
				Person.lastName 	 		& String & \parbox[t]{20cm}{\par \strut} \\
				Comment.creationDate 	 	& DateTime & \parbox[t]{20cm}{\par \strut} \\
				Comment.id 	 				& ID & \parbox[t]{20cm}{\par \strut} \\
				Comment.content 	 		& String & \parbox[t]{20cm}{\par \strut} \\
			\end{tabular}		
	\end{itemize}
		
	\item Recent posts and comments by friends or friends of friends
	\begin{itemize}
		\item \textbf{Description:}
            Given a start Person, find (most recent) Comments that are replies
            to Posts/Comments of the start Person. Only consider immediate
            (1-hop) replies, not the transitive (multi-hop) case.  Return the
            top 20 reply Comments, and the Person that created each reply
            Comment. Sort results descending by creation date of reply
            Comment, and then ascending by identifier of reply Comment.
		\item \textbf{Parameters:} \\
			\begin{tabular}{lll}
				Person.id 	 						& ID & \parbox[t]{20cm}{\par \strut} \\
				date 		 						& Date & \parbox[t]{20cm}{\par \strut} \\
			\end{tabular}		
		\item \textbf{Results:} \\
			\begin{tabular}{lll}
				Person.id 	 								& ID & \parbox[t]{20cm}{\par \strut} \\
				Person.firstName 	 						& String & \parbox[t]{20cm}{\par \strut} \\
				Person.lastName 	 						& String & \parbox[t]{20cm}{\par \strut} \\
				Post.id/Comment.id 	 						& ID & \parbox[t]{20cm}{\par \strut} \\
				Post.content/Post.imageFile/Comment.content	& String & \parbox[t]{20cm}{\par \strut} \\
				Post.creationDate/Comment.creationDate 	 	& DateTime & \parbox[t]{20cm}{\par \strut} \\
			\end{tabular}		
	\end{itemize}

	\item Friend recommendation
	\begin{itemize}
		\item \textbf{Description:}
			Given a start Person, find that Person's friends of friends (excluding start Person, and immediate friends), who were born on or after the 21st of a given month (in any year) and before the 22nd of the following month.
			Calculate the similarity between each of these Persons and start Person, where similarity for any Person is defined as follows: 
			\begin{itemize}
				\item common = number of Posts created by that Person, such that the Post has a Tag that start Person is Interested in
				\item uncommon = number of Posts created by that Person, such that the Post has no Tag that start Person is Interested in
				\item similarity = common - uncommon
			\end{itemize}
			Return top 10 Persons, their Place, and their similarity score.
			Sort results descending by similarity score, and then ascending by Person identifier
		\item \textbf{Parameters:} \\
			\begin{tabular}{lll}
				Person.id 	 						& ID & \parbox[t]{20cm}{\par \strut} \\
				month 		 						& 32-bit Integer & \parbox[t]{20cm}{// between 1-12\par \strut} \\
			\end{tabular}		
		\item \textbf{Results:} \\
			\begin{tabular}{lll}
				Person.id 	 						& ID & \parbox[t]{20cm}{\par \strut} \\
				Person.firstName 	 				& String & \parbox[t]{20cm}{\par \strut} \\
				Person.lastName 	 				& String & \parbox[t]{20cm}{\par \strut} \\
				Person.gender 	 					& String & \parbox[t]{20cm}{\par \strut} \\
				Person-isLocatedIn->Place.name 	    & Sting & \parbox[t]{20cm}{\par \strut} \\
				similarity 	 						& 32-bit Integer & \parbox[t]{20cm}{\par \strut} \\
			\end{tabular}		
	\end{itemize}
		
	\item Job referral
	\begin{itemize}
		\item \textbf{Description:}
			Given a start Person, find that Person's friends and friends of friends (excluding start Person) who started Working in some Company in a given Country, before a given date (year).
			Return top 10 Persons, the Company they worked at, and the year they started working at that Company.
			Sort results ascending by the start date, then ascending by Person identifier, and lastly by Organization name descending.
		\item \textbf{Parameters:} \\
			\begin{tabular}{lll}
				Person.id 	 				& ID & \parbox[t]{20cm}{\par \strut} \\
				Country.name 	 			& String & \parbox[t]{20cm}{\par \strut} \\
				year 		 				& 32-bit Integer & \parbox[t]{20cm}{\par \strut} \\
			\end{tabular}		
		\item \textbf{Results:} \\
			\begin{tabular}{lll}
				Person.id 	 						& ID & \parbox[t]{20cm}{\par \strut} \\
				Person.firstName 	 				& String & \parbox[t]{20cm}{\par \strut} \\
				Person.lastName 	 				& String & \parbox[t]{20cm}{\par \strut} \\
				Person-worksAt->.worksFrom 	 		& 32-bit Integer & \parbox[t]{20cm}{\par \strut} \\
				Person-worksAt->Organization.name 	& String & \parbox[t]{20cm}{\par \strut} \\
			\end{tabular}		
	\end{itemize}
		
	\item Expert search
	\begin{itemize}
		\item \textbf{Description:}
            Given a start Person, find the Comments that this Person's friends
            made in reply to Posts, considering only those Comments that are
            immediate (1-hop) replies to Posts, not the transitive (multi-hop) case.
            Only consider Posts with a Tag in a given TagClass or in a
            descendent of that TagClass.  Count the number of these reply
            Comments, and collect the Tags that were attached to the Posts they
            replied to.  Return top 20 Persons, the reply count, and the
            collection of Tags.  Sort results descending by Comment count, and
            then ascending by Person identifier
		\item \textbf{Parameters:} \\
			\begin{tabular}{lll}
				Person.id 	 					& ID & \parbox[t]{20cm}{\par \strut} \\
				TagClass.id 	 				& ID & \parbox[t]{20cm}{\par \strut} \\
			\end{tabular}		
		\item \textbf{Results:} \\
			\begin{tabular}{lll}
				Person.id 	 			& ID & \parbox[t]{20cm}{\par \strut} \\
				Person.firstName 	 	& String & \parbox[t]{20cm}{\par \strut} \\
				Person.lastName 	 	& String & \parbox[t]{20cm}{\par \strut} \\
				\{Tag.name\} 	 			& \{String\} & \parbox[t]{20cm}{\par \strut} \\
				count 	 				& 32-bit Integer & \parbox[t]{20cm}{// number of reply Comments\par \strut} \\
			\end{tabular}		
	\end{itemize}
		
	\item Single shortest path
	\begin{itemize}
		\item \textbf{Description:}
			Given two Persons, find the shortest path between these two Persons in the subgraph induced by the Knows relationships.
			Return the length of this path. 
            \begin{itemize}
                \item -1 : no path found
                \item 0: start person = end person
                \item > 0: regular case
            \end{itemize}
		\item \textbf{Parameters:} \\
			\begin{tabular}{lll}
				Person.id 	 			& ID & \parbox[t]{20cm}{// person 1\strut} \\
				Person.id 	 			& ID & \parbox[t]{20cm}{// person 2\strut} \\
			\end{tabular}		
		\item \textbf{Results:} \\
			\begin{tabular}{lll}
				length 	 			& 32-bit Integer & \parbox[t]{20cm}{\par \strut} \\
			\end{tabular}		
	\end{itemize}
		
	\item Weighted paths
	\begin{itemize}
		\item \textbf{Description:}
            Given two Persons, find all weighted paths of the shortest length
            between these two Persons in the subgraph induced by the Knows
            relationship.
            The nodes in the path are Persons.
            Weight of a path is sum of weights between every pair of consecutive Person nodes in the path.
            The weight for a pair of Persons is calculated such that every
            reply (by one of the Persons) to a Post (by the other Person)
            contributes 1.0, and every reply (by ones of the Persons) to a
            Comment (by the other Person) contributes 0.5.
            In the unlikely case that start and end are the same Person, weight is 0.
            Return all the paths with shortest length, and their weights.
            Sort results descending by path weight. Ordering of paths of the same length is not specified.
		\item \textbf{Parameters:} \\
			\begin{tabular}{lll}
				Person.id 	 			& ID & \parbox[t]{20cm}{// person 1\strut} \\
				Person.id 	 			& ID & \parbox[t]{20cm}{// person 2\strut} \\
			\end{tabular}		
		\item \textbf{Results:} \\
			\begin{tabular}{lll}
				[Person.id] 	& [ID] & \parbox[t]{20cm}{// Identifiers representing an ordered sequence of the Persons in the path \strut} \\
				weight 	 					& 64-bit Float & \parbox[t]{20cm}{\strut} \\
			\end{tabular}		
	\end{itemize}
\end{enumerate}
}

%\subsubsection{Update Query Descriptions}
%\begin{enumerate}
%    \item Add Person 
%        \begin{itemize}
%        \item \textbf{Description:} Add a Person to the social network.
%        \item \textbf{Parameters:} \\
%			\begin{tabular}{lll}
%				Person.id 	 			& ID & \parbox[t]{20cm}{\par \strut} \\
%				Person.firstName 		& String & \parbox[t]{20cm}{\par \strut} \\
%				Person.lastName 		& String & \parbox[t]{20cm}{\par \strut} \\
%				Person.gender 		& String & \parbox[t]{20cm}{\par \strut} \\
%				Person.birthDay 		& Date & \parbox[t]{20cm}{\par \strut} \\
%				Person.creationDate     & DateTime & \parbox[t]{20cm}{\par \strut} \\
%				Person.locationIp     & String & \parbox[t]{20cm}{\par \strut} \\
%				Person.browserUsed     & String & \parbox[t]{20cm}{\par \strut} \\
%				Person-isLocatedIn->City.id 	& ID & \parbox[t]{20cm}{\par \strut} \\
%                Person.speaks 	& \{ String \} & \parbox[t]{20cm}{\par \strut} \\
%                Person.emails 	& \{ String \} & \parbox[t]{20cm}{\par \strut} \\
%                Person-hasInterest->Tag.id 	& \{ ID \} & \parbox[t]{20cm}{\par \strut} \\
%                \{ Person-studyAt->University.id, \\
%                Person-studyAt->.classYear \}  & \{ID, 32-bit Integer\} & \parbox[t]{20cm}{\par \strut} \\
%                \{ Person-workAt->Company.id, \\
%                Person-workAt->.workFrom \}  & \{ID, 32-bit Integer\} & \parbox[t]{20cm}{\par \strut} \\
%            \end{tabular}		
%    \end{itemize}
%    \item Add Friendship
%        \begin{itemize}
%        \item \textbf{Description:} Add a friendship relation to the social network
%        \item \textbf{Parameters:} \\
%			\begin{tabular}{lll}
%				Person.id 	 			& ID & \parbox[t]{20cm}{// person 1\strut} \\
%				Person.id 	 			& ID & \parbox[t]{20cm}{// person 2\strut} \\
%				Person-knows->.creationDate & DateTime & \parbox[t]{20cm}{\par \strut} \\
%            \end{tabular}		
%    \end{itemize}
%    \item Add Forum
%        \begin{itemize}
%            \item \textbf{Description:} Add a Forum to the social network.
%        \item \textbf{Parameters:} \\
%			\begin{tabular}{lll}
%				Forum.id 	 			& ID & \parbox[t]{20cm}{// person 1\strut} \\
%				Forum.title 	 			& String & \parbox[t]{20cm}{// person 2\strut} \\
%				Forum.creationDate & DateTime & \parbox[t]{20cm}{\par \strut} \\
%                Forum-hasModerator->Person.id 	& \{ ID \} & \parbox[t]{20cm}{\par \strut} \\
%                Forum-hasTag->Tag.id 	& \{ ID \} & \parbox[t]{20cm}{\par \strut} \\
%            \end{tabular}		
%    \end{itemize}
%    \item Add Forum Membership
%        \begin{itemize}
%        \item \textbf{Description:} Add a Forum membership to the social network.
%        \item \textbf{Parameters:} \\
%			\begin{tabular}{lll}
%				Person.id 	 			& ID & \parbox[t]{20cm}{\par \strut} \\
%				Person-hasMember->Forum.id 	 			& ID & \parbox[t]{20cm}{\par \strut} \\
%				Person-hasMember->.joinDate 	 		& DateTime & \parbox[t]{20cm}{\par \strut} \\
%            \end{tabular}		
%    \end{itemize}
%    \item Add Post
%        \begin{itemize}
%        \item \textbf{Description:} Add a Post to the social network.
%        \item \textbf{Parameters:} \\
%			\begin{tabular}{lll}
%				Post.id 	 			& ID & \parbox[t]{20cm}{\par \strut} \\
%                Post.imageFile 	 			& String & \parbox[t]{20cm}{\par \strut} \\
%				Post.creationDate 	 		& DateTime & \parbox[t]{20cm}{\par \strut} \\
%				Post.locationIp 	 		& String & \parbox[t]{20cm}{\par \strut} \\
%				Post.browserUsed 	 		& String & \parbox[t]{20cm}{\par \strut} \\
%				Post.language 	 		    & String & \parbox[t]{20cm}{\par \strut} \\
%				Post.content 	 		    & Text & \parbox[t]{20cm}{\par \strut} \\
%				Post.length 	 		    & 32-bit Integer & \parbox[t]{20cm}{\par \strut} \\
%                Post-hasCreator->Person.id & ID & \parbox[t]{20cm}{\par \strut} \\
%                Forum-containerOf->Post.id & ID & \parbox[t]{20cm}{\par \strut} \\
%                Post-isLocatedIn->Country.id & ID & \parbox[t]{20cm}{\par \strut} \\
%                     \{Post-hasTag->Tag.id\} & \{ID\} & \parbox[t]{20cm}{\par \strut} \\
%            \end{tabular}		
%    \end{itemize}
%    \item Add Like Post
%        \begin{itemize}
%            \item \textbf{Description:} Add a Like to a Post of the social network.
%        \item \textbf{Parameters:} \\
%			\begin{tabular}{lll}
%				Person.id 	 			& ID & \parbox[t]{20cm}{\par \strut} \\
%                Post.id 	 			& ID & \parbox[t]{20cm}{\par \strut} \\
%				Person-likes->.creationDate 	 		& DateTime & \parbox[t]{20cm}{\par \strut} \\
%            \end{tabular}		
%    \end{itemize}
%    \item Add Comment
%        \begin{itemize}
%        \item \textbf{Description:} Add a Comment replying to a Post/Comment to the social network.
%        \item \textbf{Parameters:} \\
%			\begin{tabular}{lll}
%				Comment.id 	 			& ID & \parbox[t]{20cm}{\par \strut} \\
%				Comment.creationDate 			& DateTime & \parbox[t]{20cm}{\par \strut} \\
%				Comment.locationIp 	 		& String & \parbox[t]{20cm}{\par \strut} \\
%				Comment.browserUsed 	 		& String & \parbox[t]{20cm}{\par \strut} \\
%				Comment.content 	 		    & Text & \parbox[t]{20cm}{\par \strut} \\
%				Comment.length 	 		    & 32-bit Integer & \parbox[t]{20cm}{\par \strut} \\
%                Comment-hasCreator->Person.id & ID & \parbox[t]{20cm}{\par \strut} \\
%                Comment-isLocatedIn->Country.id & ID & \parbox[t]{20cm}{\par \strut} \\
%                Comment-replyOf->Post.id & ID & \parbox[t]{20cm}{ // -1 if the comment is a reply of a comment. \strut} \\
%                Comment-replyOf->Comment.id & ID & \parbox[t]{20cm}{// -1 if the comment is a reply of a post. \strut} \\
%                     \{Comment-hasTag->Tag.id\} & \{ID\} & \parbox[t]{20cm}{\par \strut} \\
%            \end{tabular}		
%    \end{itemize}
%    \item Add Like Comment
%        \begin{itemize}
%		\item \textbf{Description: Add a Like to a Comment of the social network.}
%        \item \textbf{Parameters:} \\
%			\begin{tabular}{lll}
%				Person.id 	 			& ID & \parbox[t]{20cm}{\par \strut} \\
%                Comment.id 	 			& ID & \parbox[t]{20cm}{\par \strut} \\
%				Person-likes->.creationDate 	 		& DateTime & \parbox[t]{20cm}{\par \strut} \\
%            \end{tabular}		
%    \end{itemize}
%\end{enumerate}
%
\subsubsection{Substitution parameters}\label{section:substitution}
Together with the dataset, DATAGEN produces a set of parameters per
query type. Parameter generation is designed in such a way that for each query
type, all of the generated parameters yield similar runtime behaviour of that
query.

Specifically, the selection of parameters for a query template guarantees the following properties of the resulting queries:
\begin{enumerate}
\item[P1:] the query runtime has a bounded variance: the average runtime corresponds to the behavior of the majority of the queries
\item[P2:] the runtime distribution is stable: different samples of (\eg 10) parameter bindings used in different query streams result in an identical runtime distribution across streams
\item[P3:] the optimal logical plan (optimal operator order) of the queries is the same: this ensures that a specific query template tests the system's behavior under the well-chosen technical difficulty (\eg handling voluminous joins or proper cardinality estimation for subqueries, \etc)
\end{enumerate}


As a result, the amount of data that the query touches is roughly the
same for every parameter binding, assuming that the query optimizer figures out a
reasonable execution plan for the query. This is done to avoid bindings that
cause unexpectedly long or short runtimes of queries, or even result in a
completely different optimal execution plan. Such effects could arise due to
the data skew and correlations between values in the generated dataset.

In order to get the parameter bindings for each of the queries, we have designed a \textit{Parameter Curation} procedure that works in two stages:

\begin{enumerate}
\item for each query template for all possible parameter bindings, we determine the size of intermediate results in the {\em intended} query plan. Intermediate result size heavily influences the runtime of a query, so two queries with the same operator tree and similar intermediate result sizes at every level of this operator tree are expected to have similar runtimes. This analysis is effectively a side effect of data generation, that is we keep all the necessary counts (number of friends per user, number of posts of friends \etc) as we create the dataset.
\item then, a greedy algorithm selects (``curates'') those parameters with similar intermediate result counts from the domain of all the parameters.
\end{enumerate}

Parameter bindings are stored in the \texttt{substitution\_parameters} folder
inside the data generator directory. Each query gets its bindings in a separate
file. Every line of a parameter file is a JSON-formatted collection of
key-value pairs (name of the parameter and its value). For example, the Query 1
parameter bindings are stored in file \texttt{query\_1\_param.txt}, and one of
its lines may look like this:

\vspace{-6mm}
$$
\{\text{"PersonID"}: 1, \text{"Name"}: \text{"Lei"}, \text{"PersonURI"}: \text{"http://www.ldbc.eu/ldbc\_socialnet/1.0/data/pers1"}\}
$$

Depending on implementation, the SUT may refer to persons either by IDs
(relational and graph databases) or URIs (RDF systems), so we provide both
values for the Person parameter.  Finally, parameters for short reads are taken
from those in complex reads and updates.


\subsubsection{Load Definition}\label{section:workload}

LDBC-SNB Test Driver is in charge of the execution of the Interactive Workload.
At the begining of the execution, the Test Driver creates a query mix by
assigning to each query instance, a query issue time and a set of parameters
taken from the generated substitution parameter set described above.  

Query issue times have to be carefully assigned.  Although substitution
parameters are choosen in such a way that queries of the same type take similar
time, not all query types have the same complexity and touch the same amount of
data, which makes them to scale differently for the different scale factors.
Therefore, if all query instances, regardless of their type, are issued
at the same rate, those more complex queries will dominate the execution's
result, making faster query types purposeless. To avoid this situation, each
query type is executed at a different rate. The way the execution rate is decided,
also depends of the nature of the query: complex read, short read or update.

Update queries' issue times are taken from the update streams generated by the
data generator. These are the times where the actual even happened during the
simulation of the social network. Complex reads' times are expressed in terms
of update operations. For each complex read query type, a frequency value is
assigned which specifies the relation between the number of updates performed
per complex read.  Table~\ref{table:freqs} shows the frequenceis assigned to
each query type for SF1. The frequencies of the different scale factors can be
found in Appendix~\ref{appendix:scale_factors}.

\begin{table}[H]
\centering
    \begin{tabular}{|c|c|c|c|}
    \hline
    Query Type & freq & Query Type & freq \\ 
    \hline
    \hline
    Query 1 & 26 & Query 8 & 45 \\ 
    \hline       
    Query 2 & 37 & Query 9 & 157 \\  
    \hline        
    Query 3 & 69 & Query 10 & 30 \\ 
    \hline        
    Query 4 & 36 & Query 11 & 16 \\ 
    \hline        
    Query 5 & 57 & Query 12 & 44 \\ 
    \hline        
    Query 6 & 129 & Query 13 & 19 \\  
    \hline        
    Query 7 & 87 & Query 14 & 49 \\ 
    \hline
    \end{tabular}
    \caption{Frequenceis for each query type for SF1.}
    \label{table:freqs}
\end{table}

Finally, short reads are inserted in order to balance the ration between reads
and writes, and to simulate the behavior of a real user of the social network.
For each complex read instance, a short read sequence is planned. There are two
types of short read sequences: Person centric and Message centric. Depending on
the type of the complex read, one of them is choosen. Each sequence consists of
a set of short reads which are issued in a row. The issue time assigned to each
short read in the sequence is determined at run time, and is based on the
completion time of the complex read it depends on. Once a short read sequence
is issued, there is a probability that a new short read sequence is issued.
This probability decreases as long as the number of sequences issued is
increased. Since the same random number generator seed is used across
executions, the workload is deterministic.


The specified frequencies, implicitly define the query ratios between queries
of different types, as well as a default target throughput. However the Test
Sponsor may specify a different target throughput to test,  by `squeezing''
together or ``stretching'' further apart the queries of the workload. This is
achieved  by means of the ``Time Compression Ratio'' that is multiplied by the
frequencies (see \ref{ssub:general_driver_properties}).  Therefore, different
throughputs can be tested while maintaining the relative ratios between the
different query types.


