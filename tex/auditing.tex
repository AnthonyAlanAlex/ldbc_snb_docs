

This chapter describes the rules to audit benchmark runs, that is, what
techniques are allowed and what are not, what must be provided to the auditor
and guidelines for the auditors to perform the audit.

\section{Preparation}

The first step when doing an audit is to determine the versions of the
following items that will be used for the benchmark:

\begin{itemize}
  \item The benchmark specification
  \item The data generator
  \item The driver
\end{itemize}

These must be reported in the full disclosure report to guarantee that the
benchmark run can be reproduced exactly in the future. Similarly, the test
sponsor will inform the auditor the scale factor to test. Finally, a clean test
system with enough space to store the scale factor must be provided, including
the update streams and substitution parameters.


\subsection{Collect System Details}

The next step is to collect the technical and pricing details of the system
under test. This includes the following items:

\begin{itemize}
\item Common name of the system, \eg Dell PowerEdge xxxx.
\item Type and number of CPUs, cores/threads per CPU, clock frequency and cache hierarchy characteristics (levels, size per level, \etc).
\item The amount of the system's memory, type and frequency.
\item The disk controller or motherboard type if disk controller is on the motherboard.
\item For each distinct type of secondary storage device, the number and characteristics of the device.
\item The number and type of network controllers.
\item The number and type of network switches. Wiring must be disclosed.
\item Date of availability of the system.
\end{itemize}

Only the network switches and interfaces that participate in the run need to be
reported. If the benchmark execution is entirely contained on a single machine,
no network need be reported.  The price of the hardware in question must be
disclosed and should reflect the single quantity list price that any buyer
could expect when purchasing one system with the given specification. The price
may be either an item by item price or a package price if the system is sold as
a package

Besides hardware characteristics, also software details must be collected:

\begin{itemize}
\item The DBMS and operating system name and versions.
\item Installation and configuration information of both the DBMS and operating
system, which must be provided by the test sponsor.
\item Price of the software license used, which can be tied to the number of
concurrent users or size of data.
\item Date of availability of the software.
\end{itemize}

Also, the test sponsor must provide all the source code relevant to the
benchmark.

\subsection{Setup the Benchmark Environment}

Once all the information has been collected, the auditor will setup the
environment to perform the benchmark run. This setup includes configuring the
following items:


\begin{itemize}
\item Setup the LDBC Data generator in the test machine if datasets are not
available from a trusted source.
\item Setup the LDBC driver with the connectors provided by the test sponsor.
The test sponsor must provide the configuration parameters to configure the
driver (tcr, number of threads, \etc).  ldbc.snb.interactive.update\_interleave
driver parameter must come from the updateStream.properties file, which is
created by the data generator.That parameter should never be set manually.
Also, make sure that the \verb|-rl/--results_log| is enabled.  Make sure that
all operations are enabled and the frequencies are those for the selected scale
factor. These can found in Appendix~\ref{appendix:scale_factors}.  If the driver
will be executed on a separate machine, gather the characteristics of that
machine in the same way as specified above.
\end{itemize}


\subsection{Load Data}

The test sponsor must provide all the necessary documentation and scripts
to load the dataset into the database to test. The system under test must
support the different data types needed by the benchmark for each of the
attributes at their specified precision. No data can be filtered out, everything
must be loaded.  The test sponsor must provide a tool to perform arbitrary
checks of the data or a shell to issue queries in a declarative language if the
system supports it. The auditor will measure the time to load the data, which
will be disclosed.

\section{Running the Benchmark}

Running the benchmark consists of three separate parts: (1)~validating the query
implementations, (2)~warming the database and (3)~performing the benchmark run. The
queries are validated by means of the official validation datasets provided by
LDBC consortium in their official software repositories. The auditor must load
the provided dataset and run the driver in validation mode, which will test
that the queries provide the official results.

The warmup can be performed either using the LDBC driver or externally, and the
way it is performed must be disclosed.

A valid benchmark run must last at least 2 hours of simulation time (datagen
time).  Also, in order to be valid, a benchmark run needs to meet the following
requirements. The \texttt{results\_log.csv} file contains the $\mathit{actual\_start\_time}$ and the $\mathit{scheduled\_start\_time}$ of each of the issued queries.  In order to have a valid
run, 95\% of the queries must meet the following condition:

\begin{equation*}
\mathit{actual\_start\_time} - \mathit{scheduled\_start\_time} < 1\
\mathrm{second}
\end{equation*}

If the execution of the benchmark is valid, the auditor must retrieve all the
files from directory specified by \verb|-rd/--results_dir| which includes
configuration settings used, results log, results summary, which will be
disclosed.

\section{Recovery}

Once an official run has been validated, the recovery capabilities of the
system must be tested. The system and the driver must be configured in the
same way as in during the benchmark execution. After a warmup period, an
execution of the benchmark will be performed under the same terms as in the
previous measured run.

At an arbitrary point close to 2 hours of simulation execution time, the machine
will be disconnected.  Then, the auditor will restart the database system and
will check that the last commited update (in the driver log file) is actually
in the database. The auditor will measure the time taken by the system to
recover from the failure. Also, all the information about how durability is
ensured must be disclosed. If checkpoints are used, these must be performed
with a period of 10 minutes at most.


\section{Serializability}

Optionally, the test sponsor can execute update queries atomically. The auditor
will verify that serializability is guaranteed.
