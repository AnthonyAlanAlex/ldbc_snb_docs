\chapter*{Definitions}

%{\flushleft \textbf{ACID}}: The transactional properties of Atomicity,
%Consistency, Isolation and Durability.
%
%
%{\flushleft \textbf{Commit:}} a control operation that:
%        \begin{itemize}
%            \item Is initiated by a unit of work (a Transaction) 
%            \item Is implemented by the DBMS 
%            \item Signifies that the unit of work has completed successfully
%and all tentatively modified data are to persist (until modified by some other
%operation or unit of work) 
 Upon successful completion of this control
%operation both the Transaction and the data are said to be Committed. 
%        \end{itemize}
%
%
%{\flushleft \textbf{DBMS:}} A Data Base Management System is a collection
%of programs that enable you to store, modify, and extract information from a
%database.
%
%
%
%{\flushleft \textbf{Durability:}} In general, state that persists
%across failures is said to be Durable and an implementation that ensures state
%persists across failures is said to provide Durability. In the context of the
%benchmark, Durability is more tightly defined as the SUT‘s ability to ensure
%all Committed data persist across any Single Point of Failure.
%
%{\flushleft \textbf{Measurement Window:}} This is the time window when the
%benchmark records statistics. It must fulfill the requirements defined in 
%\alert{Section XX}.
%
%{\flushleft \textbf{Performance Metric:}} The \ldbcsnb Reported Throughput as
%expressed in tps. This is known as the Performance Metric.
%
%{\flushleft \textbf{Price/Performance Metric:}} The \ldbcsnb total 3-year
%pricing divided by the Reported Throughput is price/tpsE. This is also known as
%the Price/Performance Metric.


{\flushleft \textbf{\datagen:}} Is the data generator provided by the \ldbcsnb, which
is responsible for generating the data needed to run the benchmark.

{\flushleft \textbf{DBMS:}} A DataBase Management System. 

{\flushleft \textbf{\ldbcsnb:}} Linked Data Benchmark Council Social Network Benchmark. 

{\flushleft \textbf{Query Mix:}} Refers to the ratio between read and update queries
of a workload, and the frequency at which they are issued.

{\flushleft \textbf{SF (Scale Factor):}} The \ldbcsnb is designed to target systems of
different size and scale. The scale factor determines the size of the data used
to run the benchmark, measured in Gigabytes.


{\flushleft \textbf{SUT:}} The System Under Test  is defined
to be the database system where the benchmark is executed.


{\flushleft \textbf{Test Driver:}}  A program provided by the \ldbcsnb, which
is responsible for executing the different workloads and gathering the results.

{\flushleft \textbf{Full Disclosure Report (FDR):}} The FDR is a document which allows reproduction of any benchmark result by a third party. This contains complete description of the SUT and the circumstances of the benchmark run, \eg configuration of SUT, dataset and test driver, \etc

{\flushleft \textbf{Test Sponsor:}} The Test Sponsor is the company officially
submitting the Result with the FDR and will be charged the filing fee. Although
multiple companies may sponsor a Result together, for the purposes of the LDBC
processes the Test Sponsor must be a single company. A Test Sponsor need not be
a LDBC member. The Test Sponsor is responsible for maintaining the FDR with any
necessary updates or corrections. The Test Sponsor is also the name used to
identify the Result.

%{\flushleft \textbf{Test Run:}} The entire period of time during which Drivers
%submit and the SUT completes Transactions other than Trade-Cleanup.
%
%{\flushleft \textbf{Transaction:}} 
A Database Transaction is an ACID unit of work.


%{\flushleft \textbf{Valid Transaction:}} The term Valid Transaction refers to
%any Transaction for which input data has been sent in full by the Driver, whose
%processing has been successfully completed on the SUT and whose correct output
%data has been received in full by the Driver.

{\flushleft \textbf{Workload:}} A workload refers to a set of queries of a given nature
(\ie interactive, analytical, business), how they are issued and at which rate.
