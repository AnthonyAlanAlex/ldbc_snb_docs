Together with the dataset, \datagen produces a set of parameters per
query type. Parameter generation is designed in such a way that for each query
type, all of the generated parameters yield similar runtime behaviour of that
query.

Specifically, the selection of parameters for a query template guarantees the following properties of the resulting queries:
\begin{enumerate}
\item[P1:] the query runtime has a bounded variance: the average runtime corresponds to the behavior of the majority of the queries
\item[P2:] the runtime distribution is stable: different samples of (\eg 10) parameter bindings used in different query streams result in an identical runtime distribution across streams
\item[P3:] the optimal logical plan (optimal operator order) of the queries is the same: this ensures that a specific query template tests the system's behavior under the well-chosen technical difficulty (\eg handling voluminous joins or proper cardinality estimation for subqueries, \etc)
\end{enumerate}


As a result, the amount of data that the query touches is roughly the
same for every parameter binding, assuming that the query optimizer figures out a
reasonable execution plan for the query. This is done to avoid bindings that
cause unexpectedly long or short runtimes of queries, or even result in a
completely different optimal execution plan. Such effects could arise due to
the data skew and correlations between values in the generated dataset.

In order to get the parameter bindings for each of the queries, we have designed a \textit{Parameter Curation} procedure that works in two stages:

\begin{enumerate}
\item for each query template for all possible parameter bindings, we determine the size of intermediate results in the {\em intended} query plan. Intermediate result size heavily influences the runtime of a query, so two queries with the same operator tree and similar intermediate result sizes at every level of this operator tree are expected to have similar runtimes. This analysis is effectively a side effect of data generation, that is we keep all the necessary counts (number of friends per user, number of posts of friends \etc) as we create the dataset.
\item then, a greedy algorithm selects (``curates'') those parameters with similar intermediate result counts from the domain of all the parameters.
\end{enumerate}

Parameter bindings are stored in the \texttt{substitution\_parameters} folder
inside the data generator directory. Each query gets its bindings in a separate
file. Every line of a parameter file is a JSON-formatted collection of
key-value pairs (name of the parameter and its value). For example, the Query 1
parameter bindings are stored in file \texttt{query\_1\_param.txt}, and one of
its lines may look like this:

\begin{lstlisting}
{"PersonID": 1, "Name": "Lei", "PersonURI": "http://www.ldbc.eu/ldbc_socialnet/1.0/data/pers1"}
\end{lstlisting}

Depending on implementation, the SUT may refer to persons either by IDs
(relational and graph databases) or URIs (RDF systems), so we provide both
values for the Person parameter.  Finally, parameters for short reads are taken
from those in complex reads and updates.
