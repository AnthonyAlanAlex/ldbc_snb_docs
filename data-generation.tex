\section{Data generation}
\label{section:data_generation}

\datagen uses Hadoop to implement the data generation. Detailed instructions
to configure hadoop for running \datagen can be found at the \datagen project
software repository\footnote{ \datagen repository:
https://github.com/ldbc/ldbc\_snb\_datagen}. 

\subsection{\datagen Configuration, Compilation and Execution}

\datagen is designed to be as easy to configure, compile and execute as
possible. With this objective in mind, a \texttt{run.sh} script is provided,
which handles all the compilation and execution processes.  \texttt{run.sh}
is found in the \datagen root folder. \datagen uses Apache Maven to download any
required dependencies and compile the sources.  \texttt{run.sh} needs to be
configured with two variables pointing to the proper folders. The following is
the list of variables to set:

\begin{itemize}
    \item \textbf{HADOOP\_HOME}: Points to your hadoop root folder.
    \item \textbf{LDBC\_SNB\_DATAGEN\_HOME}: Points to your \datagen root folder.
\end{itemize}

Once these variables are properly set, by typing:

\begin{lstlisting}[backgroundcolor=\color{gray}, frame=single, language=bash]
$ ./run.sh
\end{lstlisting}

\datagen is compiled and executed, and a dataset with the default options is
generated in the current folder.

A file \texttt{params.ini} is used to change the characteristics of the
generated network, as well as to set other options. \autoref{table:sndg_options}
summarizes the different available options and their default values:

\begin{table}[h]
\centering
\begin{tabular}{|p{2.5cm}|p{2.5cm}|p{10.5cm}|}
    \hline
    \textbf{Option} & \textbf{Default} & \textbf{Description} \\
    \hline
    scaleFactor & 1 & The scale factor of the data to generate. Possible values are: 1, 3, 10, 30, 100, 300 and 1000 \\
    \hline
    serializer & csv & The format of the output data. Options are: csv, csv\_merge\_foreign, ttl \\
    \hline
    compressed & false & Specifies to compress the output data in gzip. \\
    \hline
    outputDir & ./ & Specifies the folder to output the data. \\
    \hline
    numThreads & 1 & Sets the number of threads to use. Only works for pseudo-distributed mode \\
    \hline
    updateStreams & false & Sets \datagen to generate the update streams used by the driver to issue update queries. \\
    \hline
    numUpdatePartitions & 1 & The number of update stream partitions per thread. \\
    \hline
\end{tabular}
\caption{Description of the data types.}
\label{table:sndg_options}
\end{table}

An example of \texttt{params.ini} for scale factor 30, ttl serializer, 4 threads,
compressed output and a custom output dir, should look like:

\begin{lstlisting}[backgroundcolor=\color{gray},frame=single]
scaleFactor:30
serializer:ttl
compressed:true
outputDir:/home/user/output
numThreads:4
\end{lstlisting}

\datagen outputs data into HDFS. The \texttt{outputDir} directory specified in
\texttt{params.ini} file, will be automatically created in HDFS if it does not
exist. If there is not an HDFS file system mounted, then data is output to your
local file system. 

\datagen creates two directories in the \texttt{outputDir} directory: \texttt{hadoop} and
\texttt{social\_network}.  The former is used to store temporary files used during the
data generation process. The later stores the actual dataset created in the
specified format.

Additionally, \datagen creates a third directory in the \datagen home folder
called \texttt{substitution\_parameters}.  This folder contains the files with
the substitution parameters used by the driver to execute the workload.

