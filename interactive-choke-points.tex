\section{Interactive Choke Points}

\todo{See also \url{http://wiki.ldbcouncil.org/display/TUC/Interactive+Choke+Points}}

The design of the interactive workload queries has been conceived around two
main aspects: realism and technological relevance.  While realism has been
assessed by looking at existing social networks and thinking about what
interesting functionalities a user might desire from them, technological
relevance has been achieved by identifying a set of choke points queries should
stress.  These choke points capture those critical operations, techniques or
technologies that  could significatively affect the performance of the queries.
The choke points can be summarized in the following list:

\subsection*{Aggregation Performance / CP1-AggregationPerformance. Performance of aggregate calculations.}

The queries generally have a top-$k$ order by and often a group by in
addition to this.  These offer multiple optimization opportunities.
The queries also often have distinct operators, \ie distinct friends
within two steps.  Collectively these are all set operations that may
be implemented with some combination of hash and sorting, possibly
exploiting ordering in the data itself.  The aggregates are not
limited to counts and sums.  For example string concatenation occurs
as an aggregate, testing possible user defined aggregate support.
There is a wide range of cardinalities in grouping, from low, \eg country, to high, \eg post.

\subsection*{Join Performance / CP2-JoinPerformance. Voluminous joins, with or without selections.}

Each graph traversal step is in principle a join.  The join patterns are
diverse, exercising both index- and hash-based operators.   Queries are designed
so as to reward judicious use of hash join by having patterns starting with one
entity, fetching many related entities and then testing how these are related
to a third entity, \eg posts of a user with a tag of a given type.

\subsection*{Data Access Locality / CP3-DataAccessLocality. Non-full-scan access to (correlated) table data.}

Graph problems are notoriously non-local.  However, when queries touch
any non-trivial fraction of a dataset, locality will emerge and can be
exploited, for example by vectored index access or by ordering data so
that that a merge join is possible.

\subsection*{Expression Calculation / CP4-ExpressionCalculation. Efficiency in evaluating (complex) expressions.}

Queries often have expressions, including conditional expressions.
This provides opportunities for vectoring and tests efficient
management of intermediate results.

\subsection*{Correlated Subqueries / CP5-CorrelatedSubqueries. Efficiently handling dependent subqueries.}

The workload has many correlated subqueries, for example constructts
like x within two steps but not in one step, which would typically be
a correlated subquery with NOT EXISTS.  There are also scalar
subqueries with aggregation, for example returning the count of posts
satisfying a certain criteria.


\subsection*{Parallelism and Concurrency / CP6-Parallelism and Concurrency. Making use of parallel computing resources.}

All queries offer opportunities for parallelism.  This tests a wide
range of constructs, for example partitioned parallel variants of
group by and distinct.  An interactive workload will typically avoid
trivially parallelizable table scans.  Thus the opportunities that
exist must be captured by index based, navigational query plans.  The
choice of whether to parallelize or not is often left to run time and
will have to depend on the actual data seen in the execution, as
starting a parallel thread with too little to do is
counter-productive.


\subsection*{Graph Specifics / CP7 RDF and Graph Specifics.}

Graph problems are generally characterized by transitive properties
and the fact that neighboring vertices often have a large overlap in
their environments.  This makes cardinality estimation harder.  For
example, a query optimizer needs to recognize whether a relationship
has a tree or graph shape in order to make correct cardinality
estimations.  Further, there are problems aggregating properties over
a set of consecutive edges.  The workload contains business questions
dealing with paths and aggregates across paths, as well as the easier
case of determining a membership in a hierarchy with a transitive
part-of relation.
