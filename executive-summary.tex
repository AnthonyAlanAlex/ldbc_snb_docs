\execSummary{

The new data economy era, based on complexly structured, distributed and large
datasets, has brought on new demands on data management and analytics.  As a
consequence, new industry actors have appeared, offering technologies specially
built for the management of graph-like data. Also, traditional database
technologies, such as relational databases, are being adapted to the new
demands to remain competitive.

LDBC's Social Network Benchmark (\ldbcsnb) is an industrial and academic
initiative, formed by principal actors in the field of graph-like data
management. Its goal is to define a framework where different graph based
technologies can be fairly tested and compared, that can drive the
identification of systems' bottlenecks and required functionalities, and can
help researchers to open new research frontiers.

The philosophy around which \ldbcsnb is designed is to be easy to
understand, flexible and cheap to adopt. For all these reasons,
\ldbcsnb will propose different workloads representing all the usage scenarios
of graph-like database technologies, hence, targeting systems of different
nature and characteristics.  In order increase its adoption by industry and
research institutions, \ldbcsnb provides all necessary software, which are
designed to be easy to use and deploy at a small cost.

This document contains:
\begin{itemize}
\item A detailed specification of the data used in the whole \ldbcsnb benchmark.
\item A detailed specification of the workloads.
\item A detailed specification of the execution rules of the benchmark.
\item A detailed specification of the auditing rules and the full disclosure
  report's required contents.
\end{itemize}
}
